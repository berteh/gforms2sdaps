    % You can structure the document using sections. You should not use
    % subsections yourself, as these are used to typeset question text.
    \section{Range Questions}

    % Lets ask some questions.
    % \singlemark creates a single range (1-5) question.
    \singlemark{How often do you use SDAPS?}{never}{daily}

    % Now we would like to ask multiple range questions that are similar. We
    % can use a markgroup environment to typeset many range questions under
    % one heading.
    \begin{markgroup}{What do you think about the following aspects of \LaTeX?}
      \markline{equation syntax}{bad}{good}
      \markline{rendered equations}{ugly}{beautiful}
      \markline{ease of use}{hard}{easy}
    \end{markgroup}

    \section{Choice Questions}
    We can also give users a question with predefined choices. Such a list
    of choices is typesetted using a tabularx environment with equally
    sized columns. Items can span multiple columns.

    \begin{choicequestion}[3]{Which of the following Open Source
                              Optical Mark Recognition software
                              packages have you heard about?}
      \choiceitem{SDAPS}
      \choicemulticolitem{2}{Auto Multiple Choice}
      \choiceitem{QueXF}

      % Insert a text field. The freeform box automatically scales horizontally
      % The first parameter is the height of the box. The second parameter
      % is the amount of columns it should span.
      \choiceitemtext{1.2cm}{2}{Other:}
    \end{choicequestion}

    % And a more compact way of doing it; similar to markgroup
    \begin{choicegroup}{Which software do you prefere for the following tasks?}
      % We have to add the possible choices at the start.
      \groupaddchoice{\LaTeX}
      \groupaddchoice{LibreOffice}
      \groupaddchoice{Microsoft Word}
      \groupaddchoice{other}

      % After that it is possible to add each question.
      \choiceline{writing letters}
      \choiceline{creating tables}
      \choiceline{typesetting equations}
    \end{choicegroup}

    \section{Freeform text fields}

    SDAPS will extract freeform textfields such as below as images and put
    these into reports. SDAPS knows whether there is writing in the box and
    how large it is.

    % This is a textbox which is at least 2cm high. It will automatically scale
    % to fill the page.
    \textbox{2cm}{Do you have any comments?}

    % Force a new page here
    \newpage
    \section{Tricks and Features}
    SDAPS can also use circular checkboxes if you prefere. Or you can use the
    {\tt multicol} package to create multi-column layouts as is done below.

    % Set checkbox style to be circular
    \def\checkboxstyle{ellipse}

   
      \singlemark{This is a range question}{lower bound}{upper bound}%
      % Note that we need the % at the end of the last line to prevent
      % LaTeX from inserting too much whitespace.
      \label{somelabel}

      As you can see, this is a multi-column layout. The {\tt markgroup} and
      {\tt choicegroup} environments may be a bit tight in this mode.

      Lets put some more questions here, just because we can.

      \begin{choicequestion}[1]{A choice question!}
        \choiceitem{first choice}
        \choiceitem{second choice}
        \choiceitem{third choice}
        \choiceitemtext{1.2cm}{1}{other:}
      \end{choicequestion}

      \singlemark{Another range question}{lower bound}{upper bound}
      This text is closer to the question compared to question~\ref{somelabel}
      because it is not starting a new paragraph.


      \textbox{3cm}{And a freeform text field}

    That's it for the multi-column part; it was fun while it lasted!

    There are some more special commands. You can draw \checkedbox{} crossed
    checkboxes, \filledbox{} filled or \correctedbox{} filled and crossed ones. Finally there is
    also the plain \checkbox*{} checkbox using {\tt \textbackslash{}checkbox*}.

    \textbox*{2cm}{And textboxes with a fixed height. This one is exactly 2\,cm high.}

    % Reset checkbox style again.
    \def\checkboxstyle{box}
